\documentclass{article}
\usepackage{layout}

\title{Mathematics for Computer Graphics: A Primer}
\date{}

\begin{document}

\maketitle

\section*{Introduction}
Computer graphics leverages a lot of mathematical techniques, particularly from linear algebra.  This short primer provides an overview of the techniques we'll commonly use.  Particular applications are reserved for the course materials.

\section*{Numbers}
We'll call a single number a \emph{scalar}, and we'll group collections of scalars into \emph{vectors}, although we'll often give a vector different names based on its usage.  For example, we'll often call a position in space a \emph{coordinate}, a \emph{location}, and even a \emph{vertex}\footnote{Although strictly speaking, that's an incorrect usage.  A vertex represented a collection of data associated with a point in space.}, but in all cases, we'll represent all the uses as a vector of some dimension. 


\end{document}